%
% spellcheck-off
%


%%%%%%%%%%%%%%%%%%%%%%%%%%%%%%%%%%%%%%%%%%%%%%%%%%%%%%%%%%%%%%%%%%%%%%%%%%%%%%%%%%%%%%%%%%%%%%%%%%%%%%%%%%%%%%%%%%%%%%%%%%%%%
% UARP MATLAB Code Style
%%%%%%%%%%%%%%%%%%%%%%%%%%%%%%%%%%%%%%%%%%%%%%%%%%%%%%%%%%%%%%%%%%%%%%%%%%%%%%%%%%%%%%%%%%%%%%%%%%%%%%%%%%%%%%%%%%%%%%%%%%%%%
% Requires \usepackage[numbered,framed]{matlab-prettifier}

\ifdarkmode
    \definecolor{matlabcommentcolor}{RGB}{181,230,29}
    \definecolor{matlabstringcolor}{RGB}{255,174,201}
    \definecolor{matlabglobalcolor}{RGB}{73,207,206}
    \definecolor{matlabkeywordcolor}{RGB}{139,166,205}
\else
    \definecolor{matlabcommentcolor}{RGB}{28,172,0}
    \definecolor{matlabstringcolor}{RGB}{170,55,241}
    \definecolor{matlabglobalcolor}{RGB}{55,170,170}
    \definecolor{matlabkeywordcolor}{RGB}{16,16,255}
\fi

% UARP Matlab Style
\lstdefinestyle{uarpmatlab}{
   language            = Matlab,
   style               = Matlab-editor,
   basicstyle          = \scriptsize\fontencoding{T1}\ttfamily,
   mlshowsectionrules  = true,
   escapechar          = `,  
   mlcommentstyle      = \color{matlabcommentcolor},
   mlsectiontitlestyle = \bfseries\color{matlabcommentcolor},
   mlstringstyle       = \color{matlabstringcolor},
   mlkeywordstyle      = \color{matlabkeywordcolor},
   showstringspaces    = false, %without this there will be a symbol in the places where there is a space
   emph                = [1]{UConfig,UARPConfig,PCIeUARPLibEnums,PCIeUARPLibName},emphstyle=[1]\color{matlabglobalcolor}, %some words to emphasise
}


%%%%%%%%%%%%%%%%%%%%%%%%%%%%%%%%%%%%%%%%%%%%%%%%%%%%%%%%%%%%%%%%%%%%%%%%%%%%%%%%%%%%%%%%%%%%%%%%%%%%%%%%%%%%%%%%%%%%%%%%%%%%%
% Eclipse Code Style
%%%%%%%%%%%%%%%%%%%%%%%%%%%%%%%%%%%%%%%%%%%%%%%%%%%%%%%%%%%%%%%%%%%%%%%%%%%%%%%%%%%%%%%%%%%%%%%%%%%%%%%%%%%%%%%%%%%%%%%%%%%%%
% Based on https://github.com/markroyer/latex-listings-eclipse

\definecolor{eclipsestringColor}{rgb}{0.16,0.00,1.00}
\definecolor{eclipsefieldColor}{rgb}{0.16,0.00,1.00}
\definecolor{eclipseannotationColor}{rgb}{0.39,0.39,0.39}
\definecolor{eclipsekeywordColor}{rgb}{0.50,0.00,0.33}
\definecolor{eclipsecommentColor}{rgb}{0.25,0.50,0.37}
\definecolor{eclipselineNumberColor}{rgb}{0.47,0.47,0.47}

\lstdefinestyle{eclipse}{
   basicstyle         = \scriptsize\fontencoding{T1}\ttfamily,
   emphstyle          = \bfseries,
   keywordstyle       = \color{eclipsekeywordColor}\bfseries,
   commentstyle       = \color{eclipsecommentColor},
   showstringspaces   = false, %without this there will be a symbol in the places where there is a 
   stringstyle        = \color{eclipsestringColor},
   numberstyle        = \color{eclipselineNumberColor},
   numbers            = left,
   frame              = single
}

\lstdefinestyle{eclipse-c}{
   style              = eclipse,
   language           = c,
   tabsize            = 4,
   % From http://tex.stackexchange.com/questions/116534/lstlisting-line-wrapping
   breaklines         = true,
   breakatwhitespace  = true,
   postbreak          = \raisebox{0ex}[0ex][0ex]{\ensuremath{\color{red}\hookrightarrow\space}},
   % Add additional keywords here
   morekeywords       = { __attribute__, bool, __asm, __irq },
   escapeinside       = {/*!}{!*/}
}


%%%%%%%%%%%%%%%%%%%%%%%%%%%%%%%%%%%%%%%%%%%%%%%%%%%%%%%%%%%%%%%%%%%%%%%%%%%%%%%%%%%%%%%%%%%%%%%%%%%%%%%%%%%%%%%%%%%%%%%%%%%%%
% Verilog Code Style
%%%%%%%%%%%%%%%%%%%%%%%%%%%%%%%%%%%%%%%%%%%%%%%%%%%%%%%%%%%%%%%%%%%%%%%%%%%%%%%%%%%%%%%%%%%%%%%%%%%%%%%%%%%%%%%%%%%%%%%%%%%%%
% Based on https://github.com/jubobs/sclang-prettifier/blob/master/sclang-prettifier.dtx
% Recommend \usepackage[lighttt]{lmodern}

\ifdarkmode
    \definecolor{verilogbasiccolor}{RGB}{255,255,255}
    \definecolor{verilogcommentcolor}{RGB}{181,230,29}
    \definecolor{verilogkeywordcolor}{RGB}{73,207,206}
    \definecolor{verilogsystemcolor}{RGB}{255,174,201}
    \definecolor{verilognumbercolor}{RGB}{255,143,102}
    \definecolor{verilogstringcolor}{RGB}{190,190,190}
    \definecolor{verilogdefinecolor}{RGB}{185,122,87}
    \definecolor{verilogoperatorcolor}{RGB}{139,166,205}
\else
    \definecolor{verilogbasiccolor}{RGB}{0,0,0}
    \definecolor{verilogcommentcolor}{RGB}{104,180,104}
    \definecolor{verilogkeywordcolor}{RGB}{49,49,255}
    \definecolor{verilogsystemcolor}{RGB}{128,0,255}
    \definecolor{verilognumbercolor}{RGB}{255,143,102}
    \definecolor{verilogstringcolor}{RGB}{160,160,160}
    \definecolor{verilogdefinecolor}{RGB}{128,64,0}
    \definecolor{verilogoperatorcolor}{RGB}{0,0,128}
\fi

% Verilog style
\lstdefinestyle{prettyverilog}{
   language           = Verilog,
   commentstyle       = \color{verilogcommentcolor},
   alsoletter         = \$'0123456789\`,
   literate           = *{+}{{\verilogColorOperator{+}}}{1}%
                         {-}{{\verilogColorOperator{-}}}{1}%
                         {@}{{\verilogColorOperator{@}}}{1}%
                         {;}{{\verilogColorOperator{;}}}{1}%
                         {*}{{\verilogColorOperator{*}}}{1}%
                         {?}{{\verilogColorOperator{?}}}{1}%
                         {:}{{\verilogColorOperator{:}}}{1}%
                         {<}{{\verilogColorOperator{<}}}{1}%
                         {>}{{\verilogColorOperator{>}}}{1}%
                         {=}{{\verilogColorOperator{=}}}{1}%
                         {!}{{\verilogColorOperator{!}}}{1}%
                         {^}{{\verilogColorOperator{^}}}{1}%
                         {|}{{\verilogColorOperator{|}}}{1}%
                         {=}{{\verilogColorOperator{=}}}{1}%
                         {[}{{\verilogColorOperator{[}}}{1}%
                         {]}{{\verilogColorOperator{]}}}{1}%
                         {(}{{\verilogColorOperator{(}}}{1}%
                         {)}{{\verilogColorOperator{)}}}{1}%
                         {,}{{\verilogColorOperator{,}}}{1}%
                         {.}{{\verilogColorOperator{.}}}{1}%
                         {~}{{\verilogColorOperator{$\sim$}}}{1}%
                         {\%}{{\verilogColorOperator{\%}}}{1}%
                         {\&}{{\verilogColorOperator{\&}}}{1}%
                         {\#}{{\verilogColorOperator{\#}}}{1}%
                         {\ /\ }{{\verilogColorOperator{\ /\ }}}{3}%
                        ,
   morestring         = [s][\color{verilogstringcolor}]{"}{"},%
   vlogdefinestyle    = \color{verilogdefinecolor},
   vlogconstantstyle  = \color{verilognumbercolor},
   vlogsystemstyle    = \color{verilogsystemcolor},
   basicstyle         = \scriptsize\fontencoding{T1}\ttfamily\color{verilogbasiccolor},
   keywordstyle       = \bfseries\color{verilogkeywordcolor},
   identifierstyle    = \color{verilogbasiccolor},
   numbers            = left,
   numbersep          = 10pt,
   tabsize            = 4,
   escapeinside       = {/*!}{!*/},
   upquote            = true,
   sensitive          = true,
   showstringspaces   = false, %without this there will be a symbol in the places where there is a space
   frame              = single
}


\makeatletter

% Language name
\newcommand\language@verilog{Verilog}
\expandafter\lst@NormedDef\expandafter\languageNormedDefd@verilog%
  \expandafter{\language@verilog}
  
% save definition of single quote for testing
\lst@SaveOutputDef{`'}\quotesngl@verilog
\lst@SaveOutputDef{``}\backtick@verilog
\lst@SaveOutputDef{`\$}\dollar@verilog

% Extract first character token in sequence and store in macro 
% firstchar@verilog, per http://tex.stackexchange.com/a/159267/21891
\newcommand\getfirstchar@verilog{}
\newcommand\getfirstchar@@verilog{}
\newcommand\firstchar@verilog{}
\def\getfirstchar@verilog#1{\getfirstchar@@verilog#1\relax}
\def\getfirstchar@@verilog#1#2\relax{\def\firstchar@verilog{#1}}

% Initially empty hook for lst
\newcommand\addedToOutput@verilog{}
\lst@AddToHook{Output}{\addedToOutput@verilog}

% The style used for constants as set in lstdefinestyle
\newcommand\constantstyle@verilog{}
\lst@Key{vlogconstantstyle}\relax%
   {\def\constantstyle@verilog{#1}}

% The style used for defines as set in lstdefinestyle
\newcommand\definestyle@verilog{}
\lst@Key{vlogdefinestyle}\relax%
   {\def\definestyle@verilog{#1}}

% The style used for defines as set in lstdefinestyle
\newcommand\systemstyle@verilog{}
\lst@Key{vlogsystemstyle}\relax%
   {\def\systemstyle@verilog{#1}}

% Counter used to check current character is a digit
\newcount\currentchar@verilog
  
% Processing macro
\newcommand\@ddedToOutput@verilog
{%
   % If we're in \lstpkg{}' processing mode...
   \ifnum\lst@mode=\lst@Pmode%
      % Save the first token in the current identifier to \@getfirstchar
      \expandafter\getfirstchar@verilog\expandafter{\the\lst@token}%
      % Check if the token is a backtick
      \expandafter\ifx\firstchar@verilog\backtick@verilog
         % If so, then this starts a define
         \let\lst@thestyle\definestyle@verilog%
      \else
         % Check if the token is a dollar
         \expandafter\ifx\firstchar@verilog\dollar@verilog
            % If so, then this starts a system command
            \let\lst@thestyle\systemstyle@verilog%
         \else
            % Check if the token starts with a single quote
            \expandafter\ifx\firstchar@verilog\quotesngl@verilog
               % If so, then this starts a constant without length
               \let\lst@thestyle\constantstyle@verilog%
            \else
               \currentchar@verilog=48
               \loop
                  \expandafter\ifnum%
                  \expandafter`\firstchar@verilog=\currentchar@verilog%
                     \let\lst@thestyle\constantstyle@verilog%
                     \let\iterate\relax%
                  \fi
                  \advance\currentchar@verilog by \@ne%
                  \unless\ifnum\currentchar@verilog>57%
               \repeat%
            \fi
         \fi
      \fi
   \fi
}

% Add processing macro only if verilog
\lst@AddToHook{PreInit}{%
  \ifx\lst@language\languageNormedDefd@verilog%
    \let\addedToOutput@verilog\@ddedToOutput@verilog%
  \fi
}

% Colour operators in literate
\newcommand{\verilogColorOperator}[1]
{%
  \ifnum\lst@mode=\lst@Pmode\relax%
   {\bfseries\textcolor{verilogoperatorcolor}{#1}}%
  \else
    #1%
  \fi
}

\makeatother
