% Packages

\documentclass[a4paper,oneside,12pt]{templates/latex/leedsthesis}

\usepackage{graphicx}
\usepackage[export]{adjustbox}
\usepackage{rotating}
\usepackage{pdfpages}

\usepackage{epsfig}
\usepackage{epstopdf}
\usepackage{array}
\usepackage{import}
\usepackage[inline]{enumitem}
\usepackage{hanging}

\usepackage[normalem]{ulem}
\usepackage{minitoc}
\usepackage{quotchap}

% UL 30 Nov 2018 pandoc puts lists in 'tightlist' command when no space between bullet points in Rmd file
\providecommand{\tightlist}{%
  \setlength{\itemsep}{0pt}\setlength{\parskip}{0pt}}

% Positioning
\usepackage{float}
$if(nobind)$
\setlength\oddsidemargin{\dimexpr(\paperwidth-\textwidth)/2 - 1in\relax}
\setlength\evensidemargin{\oddsidemargin}
$endif$

% Colours and Tables
\usepackage{colortbl}
\usepackage{calc}
\usepackage{longtable}
\usepackage{threeparttable}
\usepackage{threeparttablex}
\usepackage{booktabs}

% Bibliography
\usepackage[style=ieee,
    refsegment=chapter,
    defernumbers=true
]{biblatex}
\newcommand*{\bibtitle}{$bibliography-heading-in-pdf$}

$for(bibliography)$
\addbibresource{$bibliography$}
$endfor$

%UL use soul package for correction highlighting
\usepackage{color, soul}
\usepackage{xcolor}
\definecolor{correctioncolor}{HTML}{CCCCFF}
\sethlcolor{correctioncolor}
\newcommand{\ctext}[3][RGB]{%
  \begingroup
  \definecolor{hlcolor}{#1}{#2}\sethlcolor{hlcolor}%
  \hl{#3}%
  \endgroup
}
\soulregister\ref7
\soulregister\cite7
\soulregister\autocite7
\soulregister\textcite7
\soulregister\pageref7

$if(corrections)$
    \correctionstrue
$endif$


% UL 1 Dec 2018, fix to include code in shaded environments
\usepackage{color}
\usepackage{fancyvrb}
\newcommand{\VerbBar}{|}
\newcommand{\VERB}{\Verb[commandchars=\\\{\}]}
\DefineVerbatimEnvironment{Highlighting}{Verbatim}{commandchars=\\\{\}}
% Add ',fontsize=\small' for more characters per line
\usepackage{framed}
\definecolor{shadecolor}{RGB}{248,248,248}
\newenvironment{Shaded}{\begin{snugshade}}{\end{snugshade}}
\newcommand{\AlertTok}[1]{\textcolor[rgb]{0.94,0.16,0.16}{#1}}
\newcommand{\AnnotationTok}[1]{\textcolor[rgb]{0.56,0.35,0.01}{\textbf{\textit{#1}}}}
\newcommand{\AttributeTok}[1]{\textcolor[rgb]{0.77,0.63,0.00}{#1}}
\newcommand{\BaseNTok}[1]{\textcolor[rgb]{0.00,0.00,0.81}{#1}}
\newcommand{\BuiltInTok}[1]{#1}
\newcommand{\CharTok}[1]{\textcolor[rgb]{0.31,0.60,0.02}{#1}}
\newcommand{\CommentTok}[1]{\textcolor[rgb]{0.56,0.35,0.01}{\textit{#1}}}
\newcommand{\CommentVarTok}[1]{\textcolor[rgb]{0.56,0.35,0.01}{\textbf{\textit{#1}}}}
\newcommand{\ConstantTok}[1]{\textcolor[rgb]{0.00,0.00,0.00}{#1}}
\newcommand{\ControlFlowTok}[1]{\textcolor[rgb]{0.13,0.29,0.53}{\textbf{#1}}}
\newcommand{\DataTypeTok}[1]{\textcolor[rgb]{0.13,0.29,0.53}{#1}}
\newcommand{\DecValTok}[1]{\textcolor[rgb]{0.00,0.00,0.81}{#1}}
\newcommand{\DocumentationTok}[1]{\textcolor[rgb]{0.56,0.35,0.01}{\textbf{\textit{#1}}}}
\newcommand{\ErrorTok}[1]{\textcolor[rgb]{0.64,0.00,0.00}{\textbf{#1}}}
\newcommand{\ExtensionTok}[1]{#1}
\newcommand{\FloatTok}[1]{\textcolor[rgb]{0.00,0.00,0.81}{#1}}
\newcommand{\FunctionTok}[1]{\textcolor[rgb]{0.00,0.00,0.00}{#1}}
\newcommand{\ImportTok}[1]{#1}
\newcommand{\InformationTok}[1]{\textcolor[rgb]{0.56,0.35,0.01}{\textbf{\textit{#1}}}}
\newcommand{\KeywordTok}[1]{\textcolor[rgb]{0.13,0.29,0.53}{\textbf{#1}}}
\newcommand{\NormalTok}[1]{#1}
\newcommand{\OperatorTok}[1]{\textcolor[rgb]{0.81,0.36,0.00}{\textbf{#1}}}
\newcommand{\OtherTok}[1]{\textcolor[rgb]{0.56,0.35,0.01}{#1}}
\newcommand{\PreprocessorTok}[1]{\textcolor[rgb]{0.56,0.35,0.01}{\textit{#1}}}
\newcommand{\RegionMarkerTok}[1]{#1}
\newcommand{\SpecialCharTok}[1]{\textcolor[rgb]{0.00,0.00,0.00}{#1}}
\newcommand{\SpecialStringTok}[1]{\textcolor[rgb]{0.31,0.60,0.02}{#1}}
\newcommand{\StringTok}[1]{\textcolor[rgb]{0.31,0.60,0.02}{#1}}
\newcommand{\VariableTok}[1]{\textcolor[rgb]{0.00,0.00,0.00}{#1}}
\newcommand{\VerbatimStringTok}[1]{\textcolor[rgb]{0.31,0.60,0.02}{#1}}
\newcommand{\WarningTok}[1]{\textcolor[rgb]{0.56,0.35,0.01}{\textbf{\textit{#1}}}}

% SI Units
\usepackage{textgreek}
\usepackage{nicefrac}
\usepackage[binary-units=true]{siunitx}
\DeclareSIUnit{\sample}{S}
\DeclareSIUnit{\bitshort}{b}
\DeclareSIUnit{\sicycle}{cycle}
\DeclareSIUnit{\sicycles}{cycles}
\DeclareSIUnit{\channel}{ch}
\DeclareSIUnit{\ounce}{oz}
\DeclareSIUnit{\french}{Fr}
\DeclareSIQualifier{\peak}{peak}
\DeclareSIQualifier{\peakpeak}{pp}
\DeclareSIUnit{\unitless}{\relax}

% Dark mode >:)
\let\oldincludegraphics\includegraphics
\newif\ifdarkmode
$if(dark-mode)$
    \darkmodetrue
    \pagecolor{darkgray}
    \color{white}
    \definecolor{formalshade}{rgb}{0.23,0.23,0.39}
    \definecolor{formalstripe}{rgb}{0.95,0.95,1}
    \definecolor{roadmapred}{RGB}{168,133,136}
    \definecolor{roadmapgreen}{RGB}{148,170,150}
    \newcommand{\linkcolour}{SkyBlue}
    \renewcommand{\includegraphics}[2][]{{\setlength{\fboxsep}{0pt}\colorbox{white}{\oldincludegraphics[#1]{#2}}}}
$else$
    \darkmodefalse
    \definecolor{formalshade}{rgb}{0.95,0.95,1}
    \definecolor{formalstripe}{rgb}{0.23,0.23,0.39}
    \definecolor{roadmapred}{RGB}{247,196,200}
    \definecolor{roadmapgreen}{RGB}{217,249,220}
    \newcommand{\linkcolour}{MidnightBlue}
$endif$

% Thesis Details
\newcommand{\thesistitle}{$title$}
\newcommand{\thesisauthor}{$author$}
\newcommand{\thesisemail}{$email$}
\newcommand{\thesisdepartment}{$department$}
\newcommand{\thesisuniversity}{$university$}
\newcommand{\thesisrequirements}{$thesis-requirements$}
\newcommand{\thesisdegree}{$thesis-degree$}
\newcommand{\thesisdate}{$thesis-date$}

\title{\thesistitle}
\degree{\thesisdegree}
\degreedate{\thesisdate}
\crest{\oldincludegraphics[width=35mm]{templates/crest.pdf}}
\collegeordept{\thesisdepartment}
\university{\thesisuniversity}

\ifpdf
    \author{\Large \thesisauthor}
    \pdfcatalog {/PageMode (/UseOutlines) /OpenAction (fitbh)}
\else
    \author{\thesisauthor}
\fi

% Watermark config
\hbadness=10000
\hfuzz=50pt
\usepackage{templates/latex/style/watermark}

% Hyperlink display config
$if(hyperlinks)$

    \hypersetup{
        pdfauthor={\thesisauthor (\thesisemail)},%
        pdftitle={\thesistitle},%
        pdfsubject={PhD Thesis for \thesisauthor},%
        pdfkeywords={PhD, Thesis},%
        pdfproducer={LaTeX},%
        pdfcreator={pdfLaTeX},%
        %
        colorlinks   = true, %Colours links instead of ugly boxes
        urlcolor     = \linkcolour, %Colour for external hyperlinks
        linkcolor    = \linkcolour, %Colour of internal links
        citecolor    = \linkcolour %Colour of citations
    }
  
    % Create an environment that rotates a page in the PDF output. This does NOT rotate
    % the content so header/footer remain correct for a portrait display, it simply sets
    % the page display mode
    \newenvironment{landscape}%
        {\clearpage\pagebreak[4]\global\pdfpageattr\expandafter{\the\pdfpageattr/Rotate 90}}%
        {\clearpage\pagebreak[4]\global\pdfpageattr\expandafter{\the\pdfpageattr/Rotate 0}}%

$else$

    \hypersetup{
        pdfauthor={\thesisauthor \thesisemail},%
        pdftitle={\thesistitle},%
        pdfsubject={PhD Thesis for \thesisauthor},%
        pdfkeywords={PhD, Thesis},%
        pdfproducer={LaTeX},%
        pdfcreator={pdfLaTeX}
        colorlinks   = true, %Colours links instead of ugly boxes
    }
    \AtBeginDocument{\NoHyper}
    \def\noToCPageLink{}
    
    % For final output, the rotate-page environment does not actually rotate the page
    % to avoid binding issues. Instead it simply keeps the \clearpage and \pagebreak
    % commands to make sure the overall layout is unaffected
    \newenvironment{landscape}%
        {\clearpage\pagebreak[4]}%
        {\clearpage\pagebreak[4]}%

$endif$

% Rotate sideways figure/table depending on the current "rotate page" environment mode
\BeforeBeginEnvironment{sidewaystable}{\begin{landscape}}
\AfterEndEnvironment{sidewaystable}{\end{landscape}}
\BeforeBeginEnvironment{sidewaysfigure}{\begin{landscape}}
\AfterEndEnvironment{sidewaysfigure}{\end{landscape}}

% Include other PDFs
\usepackage{pdfpages} 

% Include package to make 22nd, 53rd, 4th, 11th, 71st, etc.
\usepackage[super,negative]{nth} % where option [super] gives superscript 'nth'

% Prevent floats from appearing outside their section and subsection
\usepackage[section]{placeins}
\let\Oldsubsection\subsection
\renewcommand{\subsection}{\FloatBarrier\Oldsubsection}

%If an acronyms file exists, include the glossaries package and acronyms file
\usepackage[toc,acronym,shortcuts]{glossaries} 
% First parameter can be changed eg to "Glossary" or something.
% Second parameter is the max length of bold terms.
\begin{mclistof}{List of Abbreviations}{3.2cm}

\item[1-D, 2-D] One- or two-dimensional, referring in this thesis to spatial dimensions in an image.

\item[Otter] One of the finest of water mammals.

\item[Hedgehog] Quite a nice prickly friend.

\end{mclistof} 

\makeglossaries
\glsenableentrycount

% Use these in titles and figure captions rather than \gls, \Gls, \glspl, or \Glspl to avoid issues with bookmarks. This will give the fully expanded version rather than the acronym.
\let\glstitle\glsentrylong
\let\glstitlepl\glsentrylongpl
\let\glstitleacr\glsentryshort
\let\glstitleacrpl\glsentryshortpl

% Highlighted quote box environment
\usepackage{framed}
\usepackage[strict]{changepage}

\newenvironment{formal}{%
    \def\FrameCommand{%
        \hspace{1pt}%
        {\color{formalstripe}\vrule width 2pt}%
        {\color{formalshade}\vrule width 4pt}%
        \colorbox{formalshade}%
    }%
    \MakeFramed{\advance\hsize-\width\FrameRestore}%
    \noindent\hspace{-4.55pt}% disable indenting first paragraph
    \begin{adjustwidth}{}{7pt}%
    \vspace{2pt}\vspace{2pt}%
}
{%
    \vspace{2pt}\end{adjustwidth}\endMakeFramed%
}

% Code Handling
\usepackage{listings}
\usepackage[numbered,framed]{matlab-prettifier}
\renewcommand\lstlistlistingname{List of Code Listings}
\usepackage[lighttt]{lmodern}

% Caption Handling
\usepackage{caption}
\usepackage[labelformat=simple]{subcaption}
\renewcommand\thesubfigure{(\alph{subfigure})}
\renewcommand\thesubtable{(\alph{subtable})}
\DeclareCaptionLabelFormat{continued}{#1~#2 (Cont.)}
\captionsetup[ContinuedFloat]{labelformat=continued}
\captionsetup{font=normalsize}

\newcommand{\phantomsubfloat}[1]{
    {% apply caption setup only temporarily
        \captionsetup[subfigure]{labelformat=empty}
        \subfloat[][]{#1}
    }%
}

%%
%% Start of Document
%%
\begin{document}
%\language{english}

\renewcommand\baselinestretch{1.2}
\baselineskip=18pt plus 1pt

% Code Formatting Styles for listings
%
% spellcheck-off
%


%%%%%%%%%%%%%%%%%%%%%%%%%%%%%%%%%%%%%%%%%%%%%%%%%%%%%%%%%%%%%%%%%%%%%%%%%%%%%%%%%%%%%%%%%%%%%%%%%%%%%%%%%%%%%%%%%%%%%%%%%%%%%
% UARP MATLAB Code Style
%%%%%%%%%%%%%%%%%%%%%%%%%%%%%%%%%%%%%%%%%%%%%%%%%%%%%%%%%%%%%%%%%%%%%%%%%%%%%%%%%%%%%%%%%%%%%%%%%%%%%%%%%%%%%%%%%%%%%%%%%%%%%
% Requires \usepackage[numbered,framed]{matlab-prettifier}

\ifdarkmode
    \definecolor{matlabcommentcolor}{RGB}{181,230,29}
    \definecolor{matlabstringcolor}{RGB}{255,174,201}
    \definecolor{matlabglobalcolor}{RGB}{73,207,206}
    \definecolor{matlabkeywordcolor}{RGB}{139,166,205}
\else
    \definecolor{matlabcommentcolor}{RGB}{28,172,0}
    \definecolor{matlabstringcolor}{RGB}{170,55,241}
    \definecolor{matlabglobalcolor}{RGB}{55,170,170}
    \definecolor{matlabkeywordcolor}{RGB}{16,16,255}
\fi

% UARP Matlab Style
\lstdefinestyle{uarpmatlab}{
   language            = Matlab,
   style               = Matlab-editor,
   basicstyle          = \scriptsize\fontencoding{T1}\ttfamily,
   mlshowsectionrules  = true,
   escapechar          = `,  
   mlcommentstyle      = \color{matlabcommentcolor},
   mlsectiontitlestyle = \bfseries\color{matlabcommentcolor},
   mlstringstyle       = \color{matlabstringcolor},
   mlkeywordstyle      = \color{matlabkeywordcolor},
   showstringspaces    = false, %without this there will be a symbol in the places where there is a space
   emph                = [1]{UConfig,UARPConfig,PCIeUARPLibEnums,PCIeUARPLibName},emphstyle=[1]\color{matlabglobalcolor}, %some words to emphasise
}


%%%%%%%%%%%%%%%%%%%%%%%%%%%%%%%%%%%%%%%%%%%%%%%%%%%%%%%%%%%%%%%%%%%%%%%%%%%%%%%%%%%%%%%%%%%%%%%%%%%%%%%%%%%%%%%%%%%%%%%%%%%%%
% Eclipse Code Style
%%%%%%%%%%%%%%%%%%%%%%%%%%%%%%%%%%%%%%%%%%%%%%%%%%%%%%%%%%%%%%%%%%%%%%%%%%%%%%%%%%%%%%%%%%%%%%%%%%%%%%%%%%%%%%%%%%%%%%%%%%%%%
% Based on https://github.com/markroyer/latex-listings-eclipse

\definecolor{eclipsestringColor}{rgb}{0.16,0.00,1.00}
\definecolor{eclipsefieldColor}{rgb}{0.16,0.00,1.00}
\definecolor{eclipseannotationColor}{rgb}{0.39,0.39,0.39}
\definecolor{eclipsekeywordColor}{rgb}{0.50,0.00,0.33}
\definecolor{eclipsecommentColor}{rgb}{0.25,0.50,0.37}
\definecolor{eclipselineNumberColor}{rgb}{0.47,0.47,0.47}

\lstdefinestyle{eclipse}{
   basicstyle         = \scriptsize\fontencoding{T1}\ttfamily,
   emphstyle          = \bfseries,
   keywordstyle       = \color{eclipsekeywordColor}\bfseries,
   commentstyle       = \color{eclipsecommentColor},
   showstringspaces   = false, %without this there will be a symbol in the places where there is a 
   stringstyle        = \color{eclipsestringColor},
   numberstyle        = \color{eclipselineNumberColor},
   numbers            = left,
   frame              = single
}

\lstdefinestyle{eclipse-c}{
   style              = eclipse,
   language           = c,
   tabsize            = 4,
   % From http://tex.stackexchange.com/questions/116534/lstlisting-line-wrapping
   breaklines         = true,
   breakatwhitespace  = true,
   postbreak          = \raisebox{0ex}[0ex][0ex]{\ensuremath{\color{red}\hookrightarrow\space}},
   % Add additional keywords here
   morekeywords       = { __attribute__, bool, __asm, __irq },
   escapeinside       = {/*!}{!*/}
}


%%%%%%%%%%%%%%%%%%%%%%%%%%%%%%%%%%%%%%%%%%%%%%%%%%%%%%%%%%%%%%%%%%%%%%%%%%%%%%%%%%%%%%%%%%%%%%%%%%%%%%%%%%%%%%%%%%%%%%%%%%%%%
% Verilog Code Style
%%%%%%%%%%%%%%%%%%%%%%%%%%%%%%%%%%%%%%%%%%%%%%%%%%%%%%%%%%%%%%%%%%%%%%%%%%%%%%%%%%%%%%%%%%%%%%%%%%%%%%%%%%%%%%%%%%%%%%%%%%%%%
% Based on https://github.com/jubobs/sclang-prettifier/blob/master/sclang-prettifier.dtx
% Recommend \usepackage[lighttt]{lmodern}

\ifdarkmode
    \definecolor{verilogbasiccolor}{RGB}{255,255,255}
    \definecolor{verilogcommentcolor}{RGB}{181,230,29}
    \definecolor{verilogkeywordcolor}{RGB}{73,207,206}
    \definecolor{verilogsystemcolor}{RGB}{255,174,201}
    \definecolor{verilognumbercolor}{RGB}{255,143,102}
    \definecolor{verilogstringcolor}{RGB}{190,190,190}
    \definecolor{verilogdefinecolor}{RGB}{185,122,87}
    \definecolor{verilogoperatorcolor}{RGB}{139,166,205}
\else
    \definecolor{verilogbasiccolor}{RGB}{0,0,0}
    \definecolor{verilogcommentcolor}{RGB}{104,180,104}
    \definecolor{verilogkeywordcolor}{RGB}{49,49,255}
    \definecolor{verilogsystemcolor}{RGB}{128,0,255}
    \definecolor{verilognumbercolor}{RGB}{255,143,102}
    \definecolor{verilogstringcolor}{RGB}{160,160,160}
    \definecolor{verilogdefinecolor}{RGB}{128,64,0}
    \definecolor{verilogoperatorcolor}{RGB}{0,0,128}
\fi

% Verilog style
\lstdefinestyle{prettyverilog}{
   language           = Verilog,
   commentstyle       = \color{verilogcommentcolor},
   alsoletter         = \$'0123456789\`,
   literate           = *{+}{{\verilogColorOperator{+}}}{1}%
                         {-}{{\verilogColorOperator{-}}}{1}%
                         {@}{{\verilogColorOperator{@}}}{1}%
                         {;}{{\verilogColorOperator{;}}}{1}%
                         {*}{{\verilogColorOperator{*}}}{1}%
                         {?}{{\verilogColorOperator{?}}}{1}%
                         {:}{{\verilogColorOperator{:}}}{1}%
                         {<}{{\verilogColorOperator{<}}}{1}%
                         {>}{{\verilogColorOperator{>}}}{1}%
                         {=}{{\verilogColorOperator{=}}}{1}%
                         {!}{{\verilogColorOperator{!}}}{1}%
                         {^}{{\verilogColorOperator{^}}}{1}%
                         {|}{{\verilogColorOperator{|}}}{1}%
                         {=}{{\verilogColorOperator{=}}}{1}%
                         {[}{{\verilogColorOperator{[}}}{1}%
                         {]}{{\verilogColorOperator{]}}}{1}%
                         {(}{{\verilogColorOperator{(}}}{1}%
                         {)}{{\verilogColorOperator{)}}}{1}%
                         {,}{{\verilogColorOperator{,}}}{1}%
                         {.}{{\verilogColorOperator{.}}}{1}%
                         {~}{{\verilogColorOperator{$\sim$}}}{1}%
                         {\%}{{\verilogColorOperator{\%}}}{1}%
                         {\&}{{\verilogColorOperator{\&}}}{1}%
                         {\#}{{\verilogColorOperator{\#}}}{1}%
                         {\ /\ }{{\verilogColorOperator{\ /\ }}}{3}%
                        ,
   morestring         = [s][\color{verilogstringcolor}]{"}{"},%
   vlogdefinestyle    = \color{verilogdefinecolor},
   vlogconstantstyle  = \color{verilognumbercolor},
   vlogsystemstyle    = \color{verilogsystemcolor},
   basicstyle         = \scriptsize\fontencoding{T1}\ttfamily\color{verilogbasiccolor},
   keywordstyle       = \bfseries\color{verilogkeywordcolor},
   identifierstyle    = \color{verilogbasiccolor},
   numbers            = left,
   numbersep          = 10pt,
   tabsize            = 4,
   escapeinside       = {/*!}{!*/},
   upquote            = true,
   sensitive          = true,
   showstringspaces   = false, %without this there will be a symbol in the places where there is a space
   frame              = single
}


\makeatletter

% Language name
\newcommand\language@verilog{Verilog}
\expandafter\lst@NormedDef\expandafter\languageNormedDefd@verilog%
  \expandafter{\language@verilog}
  
% save definition of single quote for testing
\lst@SaveOutputDef{`'}\quotesngl@verilog
\lst@SaveOutputDef{``}\backtick@verilog
\lst@SaveOutputDef{`\$}\dollar@verilog

% Extract first character token in sequence and store in macro 
% firstchar@verilog, per http://tex.stackexchange.com/a/159267/21891
\newcommand\getfirstchar@verilog{}
\newcommand\getfirstchar@@verilog{}
\newcommand\firstchar@verilog{}
\def\getfirstchar@verilog#1{\getfirstchar@@verilog#1\relax}
\def\getfirstchar@@verilog#1#2\relax{\def\firstchar@verilog{#1}}

% Initially empty hook for lst
\newcommand\addedToOutput@verilog{}
\lst@AddToHook{Output}{\addedToOutput@verilog}

% The style used for constants as set in lstdefinestyle
\newcommand\constantstyle@verilog{}
\lst@Key{vlogconstantstyle}\relax%
   {\def\constantstyle@verilog{#1}}

% The style used for defines as set in lstdefinestyle
\newcommand\definestyle@verilog{}
\lst@Key{vlogdefinestyle}\relax%
   {\def\definestyle@verilog{#1}}

% The style used for defines as set in lstdefinestyle
\newcommand\systemstyle@verilog{}
\lst@Key{vlogsystemstyle}\relax%
   {\def\systemstyle@verilog{#1}}

% Counter used to check current character is a digit
\newcount\currentchar@verilog
  
% Processing macro
\newcommand\@ddedToOutput@verilog
{%
   % If we're in \lstpkg{}' processing mode...
   \ifnum\lst@mode=\lst@Pmode%
      % Save the first token in the current identifier to \@getfirstchar
      \expandafter\getfirstchar@verilog\expandafter{\the\lst@token}%
      % Check if the token is a backtick
      \expandafter\ifx\firstchar@verilog\backtick@verilog
         % If so, then this starts a define
         \let\lst@thestyle\definestyle@verilog%
      \else
         % Check if the token is a dollar
         \expandafter\ifx\firstchar@verilog\dollar@verilog
            % If so, then this starts a system command
            \let\lst@thestyle\systemstyle@verilog%
         \else
            % Check if the token starts with a single quote
            \expandafter\ifx\firstchar@verilog\quotesngl@verilog
               % If so, then this starts a constant without length
               \let\lst@thestyle\constantstyle@verilog%
            \else
               \currentchar@verilog=48
               \loop
                  \expandafter\ifnum%
                  \expandafter`\firstchar@verilog=\currentchar@verilog%
                     \let\lst@thestyle\constantstyle@verilog%
                     \let\iterate\relax%
                  \fi
                  \advance\currentchar@verilog by \@ne%
                  \unless\ifnum\currentchar@verilog>57%
               \repeat%
            \fi
         \fi
      \fi
   \fi
}

% Add processing macro only if verilog
\lst@AddToHook{PreInit}{%
  \ifx\lst@language\languageNormedDefd@verilog%
    \let\addedToOutput@verilog\@ddedToOutput@verilog%
  \fi
}

% Colour operators in literate
\newcommand{\verilogColorOperator}[1]
{%
  \ifnum\lst@mode=\lst@Pmode\relax%
   {\bfseries\textcolor{verilogoperatorcolor}{#1}}%
  \else
    #1%
  \fi
}

\makeatother


% Title and Front Matter

\maketitle
%set the number of sectioning levels that get number and appear in the contents
\setcounter{secnumdepth}{3}
\setcounter{tocdepth}{2}
\renewcommand\thesubsubsection{\thesubsection.\Alph{subsubsection}} 

% This redefines the \paragraph command to have a newline after it.
\let\oldparagraph\paragraph
\renewcommand{\paragraph}[1]{\oldparagraph{#1}\mbox{}\\}

% Start of Front Matter
\frontmatter

$if(declaration)$
    \begin{declaration}
        $declaration$
    \end{declaration}
$endif$

$if(dedication)$
    \begin{dedication}
        $dedication$
    \end{dedication}
$endif$

$if(acknowledgement)$
    \begin{acknowledgement}
        $acknowledgement$
    \end{acknowledgement}
$endif$

$if(abstract)$
    \begin{abstractslong}
        $abstract$
    \end{abstractslong}
$endif$

\mainmatter

% Pandoc converted full body text
$body$

% References
\setlength{\baselineskip}{0pt}
{\renewcommand*\MakeUppercase[1]{#1}%
\printbibliography[heading=bibintoc,title={\bibtitle}]}

\end{document}
